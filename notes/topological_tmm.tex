\documentclass[notitlepage,nofootinbib]{revtex4-1}

\usepackage{amsmath,amssymb}
\renewcommand{\vec}[1]{\mathbf{#1}}

\newcommand{\grad}{\boldsymbol{\nabla}}
\newcommand{\divg}{\boldsymbol{\nabla}\cdot}
\newcommand{\curl}{\boldsymbol{\nabla}\times}

\begin{document}

\title{The Transfer Matrix Method for Topological Insulators}
\author{Samuel John Palmer}
\date{\today}
\maketitle

\section{Maxwell's equations for topological insulators}
Using the principle of extremal action we find a new form of the macroscopic Maxwell equations\footnote{Marie Rider, MSc report},
\begin{subequations}\begin{alignat}{2}
	\label{eqn:divgD}
	\divg \vec{D}
	&=
	\rho + 2 \alpha \epsilon_0 c \left(\grad P_3 \cdot \vec{B}\right),
	\\
	\label{eqn:curlH}
	\curl \vec{H} - \partial_t \vec{D}
	&=
	\vec{j} - \frac{2\alpha}{\mu_0c}\left(\grad P_3 \times \vec{E} \right),
	\\
	\label{eqn:curlE}
	\curl \vec{E} + \partial_t \vec{B}
	&=
	\vec{0},
	\\
	\label{eqn:divgB}
	\divg \vec{B}
	&=
	0,
\end{alignat}\end{subequations}
where $ \vec{D} = \epsilon \vec{E} $ and $ \vec{B} = \mu \vec{H} $.

\begin{itemize}
	\item Maxwell's equations are unchanged in the bulk, where $ \grad P_3 = \vec{0} $
	\item Maxwell's equations are only modified at the surfaces, where $ \grad P_3 = \delta(\vec{r}\cdot\vec{\hat{n}}) (P_{32} - P_{31}) \vec{\hat{n}} $
	\item Extending the transfer matrix method to include topological effects therefore requires changes to the boundary conditions only
	\item In the transfer matrix method, $ E_\perp $ and $ H_\perp $ are written in terms of $ E_\parallel $ and $ H_\parallel $. Therefore the transfer matrix method requires the boundary conditions for $ E_\parallel $ and $ H_\parallel $ only.
\end{itemize}

Equation \eqref{eqn:curlE} is independent of $ P_3 $, so our condition on $ E_\parallel $ is unchanged,
\begin{equation}
	\label{eqn:EBC}
	\vec{\hat{n}} \times (\vec{E}_2 - \vec{E}_1) = \vec{0}.
\end{equation}

From integration of equation \eqref{eqn:curlH},
\begin{alignat}{1}
	\label{eqn:HBC}
	\vec{\hat{n}} \times (\vec{H}_2 - \vec{H}_1)
	&=
	\vec{K} - \frac{2\alpha}{\mu_0c}(P_{32}-P_{31}) \vec{\hat{n}} \times \vec{E},
	\\
	&=
	\vec{K}
	-\frac{2\alpha}{\mu_0c}P_{32}\vec{\hat{n}}\times\vec{E_2}
	+\frac{2\alpha}{\mu_0c}P_{31}\vec{\hat{n}}\times\vec{E_1}.
\end{alignat}
From equation \eqref{eqn:EBC} we know that $ E_\parallel $ is continuous across the interface, and therefore we can choose to interchange $ \vec{\hat{n}} \times \vec{E} = \vec{\hat{n}} \times \vec{E}_1 = \vec{\hat{n}} \times \vec{E}_2 $.


\section{Transfer matrix formulation}

Transfer matrix reference\footnote{Rumpf, CEM Lecture 4}

Assuming linear, isotropic, homogeneous layers. Within each layer, we have forward and backward traveling plane wave solutions of the form
%\begin{subequations}\begin{alignat}{2}
%	\vec{E}(\vec{r},t)
%	&=
%	\vec{E_0} e^{i \vec{k} \cdot \vec{r} - i \omega t}
%	\\
%	\vec{H}(\vec{r},t)
%	&=
%	\vec{H_0} e^{i \vec{k} \cdot \vec{r} - i \omega t}
%\end{alignat}\end{subequations}
\begin{subequations}\begin{alignat}{5}
	\begin{bmatrix}
		E_{ix}(z) \\
		E_{iy}(z)
	\end{bmatrix}
	&=&
	\begin{bmatrix}
		c_{ix}^+ \\
		c_{iy}^+
	\end{bmatrix}e^{+i k_{iz} z}
	&+&
	\begin{bmatrix}
		c_{ix}^- \\
		c_{iy}^-
	\end{bmatrix}e^{-i k_{iz} z}
	\\
	\begin{bmatrix}
		\tilde{H}_{ix}(z) \\
		\tilde{H}_{iy}(z)
	\end{bmatrix}
	&=&
	\vec{V_i}\begin{bmatrix}
		c_{ix}^+ \\
		c_{iy}^+
	\end{bmatrix}e^{+i k_{iz} z}
	&-&
	\vec{V_i}\begin{bmatrix}
		c_{ix}^- \\
		c_{iy}^-
	\end{bmatrix}e^{-i k_{iz} z}
\end{alignat}\end{subequations}
\begin{equation}
	\boldsymbol{\psi_i}(z)=
	\begin{bmatrix}
		E_{ix}(z) \\
		E_{iy}(z) \\
		\tilde{H}_{ix}(z) \\
		\tilde{H}_{iy}(z)
	\end{bmatrix}
	=
	\vec{W_i} \begin{bmatrix}
		\vec{c_i^+} e^{+ i k_z z} \\
		\vec{c_i^-} e^{- i k_z z}
	\end{bmatrix}
%	=
%	\begin{bmatrix}
%		\mathbf{I} & \phantom{-}\mathbf{I} \\
%		\mathbf{V_i} & -\mathbf{V_i} \\
%	\end{bmatrix}
%	\begin{bmatrix}
%		  e^{+i k_{iz} z} \mathbf{I} & \\
%		& e^{-i k_{iz} z} \mathbf{I}
%	\end{bmatrix}
%	\begin{bmatrix}
%		c^+_{ix} \\
%		c^+_{iy} \\
%		c^-_{ix} \\
%		c^-_{iy}
%	\end{bmatrix}
\end{equation}

We seek the transfer matrix defined as $ \vec{c_{i+1}} = \vec{T_i} \vec{c_i} $. If we define\footnote{I should fix the inconsistancy of one of these acting on $c$ and the other on $\psi$}
\begin{alignat}{1}
	\vec{c_i}(L_i) &= \vec{T_i^\Phi} \vec{c_i}(0) \\
	\boldsymbol{\psi_{i+1}}(0) &= \vec{T_i^\Delta} \boldsymbol{\psi_i}(L_i)
\end{alignat}
such that $ \vec{W_{i+1}} \vec{c_{i+1}} = \vec{T_i^\Delta} \vec{W_i} \vec{T_i^\Phi} \vec{c_i}$ then the transfer matrix is
\begin{equation}
	\vec{T_i} = \vec{W_{i+1}^{-1}} \vec{T_i^\Delta} \vec{W_i} \vec{T_i^\Phi}
\end{equation}.

The matrix $ \vec{T_i^\Delta} $ is (see Appendix \ref{app:TiDelta}),
\begin{equation}
	\vec{T_i^\Delta}
	=
	\begin{bmatrix}
		1 & 0 & 0 & 0 \\
		0 & 1 & 0 & 0 \\
		-i2\alpha \Delta P_3 & 0 & 1 & 0 \\
		0 & -i2\alpha \Delta P_3 & 0 & 1
	\end{bmatrix}
\end{equation}
where $ \Delta P_3 = P_{3,i+1} - P_{3,i}? $
For $ P_{31} = P_{32} $ this reduces to the identity matrix.

The tangential components of the wavevector, $ k_x = k_0 n_1 \cos\phi \sin\theta $ and $ k_y = k_0 n_1 \dots $, are continuous across each interface. For each layer, $ i $, we calculate the normal component,
\begin{equation}
	k_{z,i} = \sqrt{(nk_0)^2 - k_x^2 -k_y^2}
\end{equation}

\appendix

\section{\label{app:TiDelta}Formula of $ \vec{T_i^\Delta} $}

Boundary condition equation \eqref{eqn:EBC} gives
\begin{equation}
	\begin{bmatrix}
		E_{i+1,x}(0) \\
		E_{i+1,y}(0)
	\end{bmatrix}
	=
	\begin{bmatrix}
		E_{i,x}(L_i) \\
		E_{i,y}(L_i)
	\end{bmatrix}
\end{equation}
and, not forgetting that we have normalised $\tilde{H} = +i \eta_0 H = i \mu_0 c H $, boundary condition equation \eqref{eqn:HBC} gives
\begin{equation}
	\begin{bmatrix}
		\tilde{H}_{i+1,x}(0) \\
		\tilde{H}_{i+1,y}(0)
	\end{bmatrix}
	+
	i2\alpha P_{3,i+1}\begin{bmatrix}
		E_{i+1,x}(0) \\
		E_{i+1,y}(0)
	\end{bmatrix}
	=
	\begin{bmatrix}
		\tilde{H}_{i,x}(L_i) \\
		\tilde{H}_{i,y}(L_i)
	\end{bmatrix}
	+
	i2\alpha P_{3,i}\begin{bmatrix}
		E_{i,x}(L_i) \\
		E_{i,y}(L_i)
	\end{bmatrix}
\end{equation}
So altogether in terms of $ \boldsymbol{\psi_i} $ we have
\begin{alignat}{1}
	\vec{T_i^\Delta}
	&=
	\begin{bmatrix}
		1 & 0 & 0 & 0 \\
		0 & 1 & 0 & 0 \\
		i2\alpha P_{3,i+1} & 0 & 1 & 0 \\
		0 & i2\alpha P_{3,i+1} & 0 & 1
	\end{bmatrix}^{-1}
	\begin{bmatrix}
		1 & 0 & 0 & 0 \\
		0 & 1 & 0 & 0 \\
		i2\alpha P_{3,i} & 0 & 1 & 0 \\
		0 & i2\alpha P_{3,i} & 0 & 1
	\end{bmatrix}
	\\
	&=
	\begin{bmatrix}
		1 & 0 & 0 & 0 \\
		0 & 1 & 0 & 0 \\
		-i2\alpha \Delta P_3 & 0 & 1 & 0 \\
		0 & -i2\alpha \Delta P_3 & 0 & 1
	\end{bmatrix}
\end{alignat}
where $ \Delta P_3 = P_{3,i+1} - P_{3,i} $.

\end{document}